% Options for packages loaded elsewhere
\PassOptionsToPackage{unicode}{hyperref}
\PassOptionsToPackage{hyphens}{url}
%
\documentclass[
]{article}
\usepackage{amsmath,amssymb}
\usepackage{lmodern}
\usepackage{ifxetex,ifluatex}
\ifnum 0\ifxetex 1\fi\ifluatex 1\fi=0 % if pdftex
  \usepackage[T1]{fontenc}
  \usepackage[utf8]{inputenc}
  \usepackage{textcomp} % provide euro and other symbols
\else % if luatex or xetex
  \usepackage{unicode-math}
  \defaultfontfeatures{Scale=MatchLowercase}
  \defaultfontfeatures[\rmfamily]{Ligatures=TeX,Scale=1}
\fi
% Use upquote if available, for straight quotes in verbatim environments
\IfFileExists{upquote.sty}{\usepackage{upquote}}{}
\IfFileExists{microtype.sty}{% use microtype if available
  \usepackage[]{microtype}
  \UseMicrotypeSet[protrusion]{basicmath} % disable protrusion for tt fonts
}{}
\makeatletter
\@ifundefined{KOMAClassName}{% if non-KOMA class
  \IfFileExists{parskip.sty}{%
    \usepackage{parskip}
  }{% else
    \setlength{\parindent}{0pt}
    \setlength{\parskip}{6pt plus 2pt minus 1pt}}
}{% if KOMA class
  \KOMAoptions{parskip=half}}
\makeatother
\usepackage{xcolor}
\IfFileExists{xurl.sty}{\usepackage{xurl}}{} % add URL line breaks if available
\IfFileExists{bookmark.sty}{\usepackage{bookmark}}{\usepackage{hyperref}}
\hypersetup{
  pdftitle={Report on Billion Price Project},
  hidelinks,
  pdfcreator={LaTeX via pandoc}}
\urlstyle{same} % disable monospaced font for URLs
\usepackage[margin=1in]{geometry}
\usepackage{graphicx}
\makeatletter
\def\maxwidth{\ifdim\Gin@nat@width>\linewidth\linewidth\else\Gin@nat@width\fi}
\def\maxheight{\ifdim\Gin@nat@height>\textheight\textheight\else\Gin@nat@height\fi}
\makeatother
% Scale images if necessary, so that they will not overflow the page
% margins by default, and it is still possible to overwrite the defaults
% using explicit options in \includegraphics[width, height, ...]{}
\setkeys{Gin}{width=\maxwidth,height=\maxheight,keepaspectratio}
% Set default figure placement to htbp
\makeatletter
\def\fps@figure{htbp}
\makeatother
\setlength{\emergencystretch}{3em} % prevent overfull lines
\providecommand{\tightlist}{%
  \setlength{\itemsep}{0pt}\setlength{\parskip}{0pt}}
\setcounter{secnumdepth}{-\maxdimen} % remove section numbering
\usepackage{float}
\usepackage{booktabs}
\usepackage{longtable}
\usepackage{array}
\usepackage{multirow}
\usepackage{wrapfig}
\usepackage{colortbl}
\usepackage{pdflscape}
\usepackage{tabu}
\usepackage{threeparttable}
\usepackage{threeparttablex}
\usepackage[normalem]{ulem}
\usepackage{makecell}
\usepackage{xcolor}
\usepackage{siunitx}
\newcolumntype{d}{S[input-symbols = ()]}
\ifluatex
  \usepackage{selnolig}  % disable illegal ligatures
\fi

\title{Report on Billion Price Project}
\author{}
\date{\vspace{-2.5em}}

\begin{document}
\maketitle

\hypertarget{introduction}{%
\subsection{Introduction}\label{introduction}}

This is a report on \emph{The Billion Price Project}.

HERE COMES THE MOTIVATION WHY THIS IS A MEANINGFUL PROJECT AND WHAT IS
THE MAIN GOAL!

For more details on the project see:
\url{http://www.thebillionpricesproject.com/}.

\hypertarget{data}{%
\subsection{Data}\label{data}}

HERE COMES A DETAILLED EXPLANATION ABOUT WHERE THE DATA COMES FROM AND
IF IT IS REPRESENTATIVE OR NOT.

Our main interest is whether online prices are lower or higher than
simple retail store prices. We investigated the data on the collected
prices and we have the following descriptive statistics on online,
in-store prices and in their differences.

\begin{table}[!h]

\caption{\label{tab:unnamed-chunk-1}Descriptive statistics of prices}
\centering
\begin{tabular}[t]{lrrrrrrr}
\toprule
  & Mean & Median & SD & Min & Max & P05 & P95\\
\midrule
Retail & \num{55.22} & \num{14.49} & \num{135.49} & \num{0.25} & \num{970.00} & \num{1.99} & \num{219.00}\\
Online & \num{54.74} & \num{13.99} & \num{133.86} & \num{0.25} & \num{970.00} & \num{1.99} & \num{225.00}\\
Price difference & \num{-0.48} & \num{0.00} & \num{16.27} & \num{-380.13} & \num{450.01} & \num{-2.30} & \num{1.34}\\
\bottomrule
\multicolumn{8}{l}{\rule{0pt}{1em}Data are available from: https://osf.io/yhbr5/}\\
\end{tabular}
\end{table}

The number of observations is 7891 for all of our key variables.

DESCRIPTION OF THE SUMMARY STATS: WHAT CAN WE LEARN FROM THEM?

As the focus is the price difference, the next Figure shows the
histogram for this variable.

\begin{center}\includegraphics{report_bpp_files/figure-latex/unnamed-chunk-2-1} \end{center}

DESCRIPTION OF THE FIGURE. WHAT DOES IT TELS US?

(May change the order of descriptive stats and graph.)

\hypertarget{testing-price-differences}{%
\subsection{Testing Price Differences}\label{testing-price-differences}}

We test the hypothesis, whether the price difference is zero, therefore
there is no difference between retail and onlin prices:

\[H_0:=\text{price online} - \text{price retail} = 0\]
\[H_A:=\text{price online} - \text{price retail} \neq 0\] Running a
two-sided t-test, we have the t-statistic as -2.62 and the p-value as
0.01. The 95\% confidence intervals are: -0.84 and -0.12. Based on these
results with 95\% confidence we can reject the hypothesis that the two
price would be the same in this particular sample.

\hypertarget{robustness-check-heterogeneity-analysis}{%
\subsection{Robustness check / `Heterogeneity
analysis'}\label{robustness-check-heterogeneity-analysis}}

Task: calculate and report t-tests for each countries.

You should report: Country, mean of p\_diff, se of the mean for p\_diff,
number of observations in each country, t-statistic, p-value.

Hint: use `kable()' and to hold the table position you can define the
following argument: `position = ``H''\,' . Take care of caption, number
of digits you use and the name of variables you report! You may check
how the output changes if you use `booktabs = TRUE' input for kable!

\begin{table}[H]

\caption{\label{tab:unnamed-chunk-4}Online and retail price differences by countries and t-tests}
\centering
\begin{tabular}[t]{lrrrrr}
\toprule
Country & Mean & SE of the Mean & Num.Obs & t-stat & p-val\\
\midrule
BRAZIL & -0.9053 & 0.7847 & 122 & -1.1537 & 0.1254\\
CHINA & -0.5105 & 0.8411 & 19 & -0.6070 & 0.2757\\
GERMANY & 3.6797 & 1.8658 & 420 & 1.9722 & 0.0246\\
JAPAN & -11.9829 & 2.1467 & 350 & -5.5820 & 0.0000\\
SOUTHAFRICA & -2.5297 & 0.8319 & 541 & -3.0408 & 0.0012\\
\addlinespace
USA & 0.0545 & 0.1246 & 6439 & 0.4372 & 0.3310\\
\bottomrule
\end{tabular}
\end{table}

Extra: In words, select those countries, where you can not reject the
alternative that the prices are different. With the command
'\textcolor{red}{this is red} you can highlight these countries!

Countries, where we can not reject the alternative with 95\% confidence
(or with 5\% significance level), that the prices are different, hence
retail and online prices might differ:
\textcolor{red}{GERMANY, JAPAN, SOUTHAFRICA}

\hypertarget{conclusion}{%
\subsection{Conclusion}\label{conclusion}}

HERE COMES WHAT WE HAVE LEARNED AND WHAT WOULD STRENGHTEN AND WEAKEN OUR
ANALYSIS.

\end{document}
